\section*{Introduction}

GSet library is a C library to manipulate sets of data.\\

Elements of the GSet are void pointers toward any kind of data. These data must be allocated and freed separately. The GSet only provides a mean to manipulate sets of pointers toward these data.\\

The GSet offers functions to add elements (at first position, last position, given position, or sorting based on a float value), to access elements (at first position, last position, given position), to get index of first/last element pointing to a given data, to remove elements (at first position, last position, given position, or first/last/all pointing toward a given data), to search for data in elements (first one or last one), to print the set on a stream, to split, merge and sort the set.\\

The library also provides two iterator structures to run through a GSet forward or backward, and apply a user defined function on each element.\\ 

It uses the \begin{ttfamily}PBErr\end{ttfamily} library.\\

\section{Interface}

\begin{scriptsize}
\begin{ttfamily}
\verbatiminput{../gset.h}
\end{ttfamily}
\end{scriptsize}

\section{Code}

\subsection{gset.c}

\begin{scriptsize}
\begin{ttfamily}
\verbatiminput{../gset.c}
\end{ttfamily}
\end{scriptsize}

\subsection{gset-inline.c}

\begin{scriptsize}
\begin{ttfamily}
\verbatiminput{../gset-inline.c}
\end{ttfamily}
\end{scriptsize}

\section{Makefile}

\begin{scriptsize}
\begin{ttfamily}
\verbatiminput{../Makefile}
\end{ttfamily}
\end{scriptsize}

\section{Unit tests}

\begin{scriptsize}
\begin{ttfamily}
\verbatiminput{../main.c}
\end{ttfamily}
\end{scriptsize}

\section{Unit tests output}

\begin{scriptsize}
\begin{ttfamily}
\verbatiminput{../unitTestRef.txt}
\end{ttfamily}
\end{scriptsize}

