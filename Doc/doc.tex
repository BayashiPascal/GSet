\documentclass[12pt, a4paper]{article}

\usepackage{amsmath}
\usepackage{amsfonts}
\usepackage{amssymb}
\usepackage{graphicx}
\usepackage{float}
\usepackage{listings}
\usepackage{rotating}
\usepackage{tikz}
\usepackage{verbatim}
\pdfgentounicode=1
\pdfmapline{+cyberb@Unicode@  <cyberbit.ttf}

\begin{document}

\title{GSet}
\author{P. Baillehache}
\date{\today}
\maketitle

\tableofcontents

\section*{Introduction}

GSet library is a C library to manipulate sets of data.\\

Elements of the GSet are void pointers toward any kind of data. These data must be allocated and freed separately. The GSet only provides a mean to manipulate sets of pointers toward these data.\\

The GSet offers functions to add elements (at first position, last position, given position, or sorting based on a float value), to access elements (at first position, last position, given position), to get index of first/last element pointing to a given data, to remove elements (at first position, last position, given position, or first/last/all pointing toward a given data), to search for data in elements (first one or last one), to print the set on a stream, to split, merge and sort the set.\\ 

\section{Interface}

\begin{scriptsize}
\begin{ttfamily}
\verbatiminput{../gset.h}
\end{ttfamily}
\end{scriptsize}

\section{Code}

\begin{scriptsize}
\begin{ttfamily}
\verbatiminput{../gset.c}
\end{ttfamily}
\end{scriptsize}


\section{Makefile}

\begin{scriptsize}
\begin{ttfamily}
\verbatiminput{../Makefile}
\end{ttfamily}
\end{scriptsize}

\section{Usage}

\begin{scriptsize}
\begin{ttfamily}
\verbatiminput{../gset_main.c}
\end{ttfamily}
\end{scriptsize}

Output:\\

\begin{scriptsize}
\begin{ttfamily}
\verbatiminput{../output.txt}
\end{ttfamily}
\end{scriptsize}

\end{document}


