\documentclass[12pt, a4paper]{article}

\usepackage{amsmath}
\usepackage{amsfonts}
\usepackage{amssymb}
\usepackage{graphicx}
\usepackage{float}
\usepackage{listings}
\usepackage{rotating}
\usepackage{tikz}
\pdfgentounicode=1
\pdfmapline{+cyberb@Unicode@  <cyberbit.ttf}

\begin{document}

\title{GSet}
\author{P. Baillehache}
\date{\today}
\maketitle

\tableofcontents

\section*{Introduction}

GSet library is a C library to manipulate sets of data.\\

Elements of the GSet are void pointers toward any kind of data. These data must be allocated and freed separately. The GSet only provides a mean to manipulate sets of pointers toward these data.\\

The GSet offers functions to add elements (at first position, last position, given position, or sorting based on a float value), to access elements (at first position, last position, given position), to get index of first/last element pointing to a given data, to remove elements (at first position, last position, given position, or first/last/all pointing toward a given data), to search for data in elements (first one or last one), to print the set on a stream.\\ 

\section{Interface}

\begin{scriptsize}
\begin{ttfamily}
\begin{lstlisting}
// *************** GSET.H ***************
#ifndef GSET_H
#define GSET_H

// ================= Include ==================
#include <stdlib.h>
#include <stdio.h>

// ================= Define ==================

// ================= Data structures ===================

// Structure of one element of the GSet
struct GSetElem;
typedef struct GSetElem {
  // Pointer toward the data
  void* _data;
  // Pointer toward the next element in the GSet
  struct GSetElem *_next;
  // Pointer toward the previous element in the GSet
  struct GSetElem *_prev;
  // Value to sort element in the GSet, 0.0 by default
  double _sortVal;
} GSetElem;

// Structure of the GSet
typedef struct GSet {
  // Pointer toward the element at the head of the GSet
  GSetElem *_head;
  // Pointer toward the last element of the GSet
  GSetElem *_tail;
  // Number of element in the GSet
  int _nbElem;
} GSet;

// ================ Functions declaration ====================

// Function to create a new GSet,
// Return a pointer toward the new GSet, or null if it couldn't
// create the GSet
GSet* GSetCreate();

// Function to clone a GSet,
// Return a pointer toward the new GSet, or null if it couldn't
// clone the GSet
GSet* GSetClone(GSet *s);

// Function to free the memory used by the GSet
void GSetFree(GSet **s);

// Function to empty the GSet
void GSetFlush(GSet *s);

// Function to print a GSet
// Use the function 'printData' to print the data pointed to by 
// the elements, and print 'sep' between each element
// Do nothing if arguments are invalid
void GSetPrint(GSet *s, FILE* stream, 
  void(*printData)(void *data, FILE *stream), char *sep);

// Function to insert an element pointing toward 'data' at the 
// head of the GSet
// Do nothing if arguments are invalid
void GSetPush(GSet *s, void* data);

// Function to insert an element pointing toward 'data' at the 
// position defined by 'v' sorting the set in decreasing order
// Do nothing if arguments are invalid
void GSetAddSort(GSet *s, void* data, double v);

// Function to insert an element pointing toward 'data' at the 
// 'iElem'-th position 
// If 'iElem' is greater than or equal to the number of element
// in the GSet, elements pointing toward null data are added
// Do nothing if arguments are invalid
void GSetInsert(GSet *s, void* data, int iElem);

// Function to insert an element pointing toward 'data' at the 
// tail of the GSet
// Do nothing if arguments are invalid
void GSetAppend(GSet *s, void* data);

// Function to remove the element at the head of the GSet
// Return the data pointed to by the removed element, or null if the 
// GSet is empty
// Return null if arguments are invalid
void* GSetPop(GSet *s);

// Function to remove the element at the tail of the GSet
// Return the data pointed to by the removed element, or null if the 
// GSet is empty
// Return null if arguments are invalid
void* GSetDrop(GSet *s);

// Function to remove the element at the 'iElem'-th position of the GSet
// Return the data pointed to by the removed element
// Return null if arguments are invalid
void* GSetRemove(GSet *s, int iElem);

// Function to remove the first element of the GSet pointing to 'data'
// Do nothing if arguments are invalid
void GSetRemoveFirst(GSet *s, void *data);

// Function to remove the last element of the GSet pointing to 'data'
// Do nothing if arguments are invalid
void GSetRemoveLast(GSet *s, void *data);

// Function to remove all the selement of the GSet pointing to 'data'
// Do nothing if arguments are invalid
void GSetRemoveAll(GSet *s, void *data);

// Function to get the element at the 'iElem'-th position of the GSet
// without removing it
// Return the data pointed to by the element
// Return null if arguments are invalid
void* GSetGet(GSet *s, int iElem);

// Function to get the index of the first element of the GSet
// which point to 'data'
// Return -1 if arguments are invalid or 'data' is not in the GSet
int GSetGetIndexFirst(GSet *s, void *data);

// Function to get the index of the last element of the GSet
// which point to 'data'
// Return -1 if arguments are invalid or 'data' is not in the GSet
int GSetGetIndexLast(GSet *s, void *data);

#endif
\end{lstlisting}
\end{ttfamily}
\end{scriptsize}

\section{Code}

\begin{scriptsize}
\begin{ttfamily}
\begin{lstlisting}
// *************** GSET.C ***************

// ================= Include ==================
#include "gset.h"

// ================ Functions implementation ==================

// Function to create a new GSet,
// Return a pointer toward the new GSet, or null if it couldn't
// create the GSet
GSet* GSetCreate() {
  // Allocate memory for the GSet
  GSet *s = (GSet*)malloc(sizeof(GSet));
  // If we couldn't allocate memory return null
  if (s == NULL) return NULL;
  // Set the pointer to head and tail, and the number of element
  s->_head = NULL;
  s->_tail = NULL;
  s->_nbElem = 0;
  // Return the new GSet
  return s;
}

// Function to clone a GSet,
// Return a pointer toward the new GSet, or null if it couldn't
// clone the GSet
GSet* GSetClone(GSet *s) {
  // If the arguments are invalid, return NULL
  if (s == NULL) return NULL;
  // Create the clone
  GSet *c = GSetCreate();
  // If the clone could be created
  if (c != NULL) {
    // Set a pointer to the head of the set
    GSetElem *ptr = s->_head;
    // While the pointer is not at the end of the set
    while (ptr != NULL) {
      // Append the data of the current pointer to the clone
      GSetAppend(c, ptr->_data);
      // Move the pointer to the next element
      ptr = ptr->_next;
    }
  }
  // Return the clone
  return c;
}

// Function to free the memory used by the GSet
void GSetFree(GSet **s) {
  // If the arguments are invalid, stop
  if (s == NULL || *s == NULL) return;
  // Empty the GSet
  GSetFlush(*s);
  // Free the memory
  free(*s);
  // Set the pointer to null
  *s = NULL;
}

// Function to empty the GSet
void GSetFlush(GSet *s) {
  // If the arguments are invalid, stop
  if (s == NULL) return;
  // Pop element until the GSet is null
  void *p = NULL;
  while (s->_nbElem > 0) p = GSetPop(s);
  // To avoid warning
  p = p;
}

// Function to print a GSet
// Use the function 'printData' to print the data pointed to by 
// the elements, and print 'sep' between each element
// If printData is null, print the pointer value instead
// Do nothing if arguments are invalid
void GSetPrint(GSet *s, FILE* stream, 
  void(*printData)(void *data, FILE *stream), char *sep) {
  // If the arguments are invalid, stop
  if (s == NULL || stream == NULL || 
    sep == NULL) return;
  // Set a pointer to the head element
  GSetElem *p = s->_head;
  // While the pointer hasn't reach the end
  while (p != NULL) {
    // If there is a print function for the data
    if (printData != NULL) {
      // Use the argument function to print the data of the 
      // current element
      (*printData)(p->_data, stream);
    // Else, there is no print function for the data
    } else {
      // Print the pointer value instead
      fprintf(stream, "%p", p->_data);
    }
    // Flush the stream
    fflush(stream);
    // Move to the next element
    p = p->_next;
    // If there is a next element
    if (p != NULL)
      // Print the separator
      fprintf(stream, "%s", sep);
  }
}

// Function to insert an element pointing toward 'data' at the 
// head of the GSet
// Do nothing if arguments are invalid
void GSetPush(GSet *s, void* data) {
  // If the arguments are invalid, stop
  if (s == NULL || data == NULL) return;
  // Allocate memory for the new element
  GSetElem *e = (GSetElem*)malloc(sizeof(GSetElem));
  // If we could allocate memory
  if (e != NULL) {
    // Memorize the pointer toward data
    e->_data = data;
    // By default set the sorting value to 0.0
    e->_sortVal = 0.0;
    // Add the element at the head of the GSet 
    e->_prev = NULL;
    if (s->_head != NULL) s->_head->_prev = e;
    e->_next = s->_head;
    s->_head = e;
    if (s->_tail == NULL) s->_tail = e;
    // Increment the number of elements in the GSet
    ++(s->_nbElem);
  }
}

// Function to insert an element pointing toward 'data' at the 
// position defined by 'v' sorting the set in decreasing order
// Do nothing if arguments are invalid
void GSetAddSort(GSet *s, void* data, double v) {
  // If the arguments are invalid, stop
  if (s == NULL || data == NULL) return;
  // Allocate memory for the new element
  GSetElem *e = (GSetElem*)malloc(sizeof(GSetElem));
  // If we could allocate memory
  if (e != NULL) {
    // Memorize the pointer toward data
    e->_data = data;
    // Memorize the sorting value
    e->_sortVal = v;
    // If the GSet is empty
    if (s->_nbElem == 0) {
      // Add the element at the head of the GSet
      s->_head = e;
      s->_tail = e;
      e->_next = NULL;
      e->_prev = NULL;
    } else {
      // Set a pointer to the head of the GSet
      GSetElem *p = s->_head;
      // While the pointed element has a greater value than the 
      // new element, move the pointer to the next element
      while (p != NULL && p->_sortVal >= v) p = p->_next;
      // Set the next element of the new element to the current element
      e->_next = p;
      // If the current element is not null
      if (p != NULL) {
        // Insert the new element inside the list of elements before p
        e->_prev = p->_prev;
        if (p->_prev != NULL) 
          p->_prev->_next = e;
        else
          s->_head = e;
        p->_prev = e;
      // Else, if the current element is null
      } else {
        // Insert the new element at the tail of the GSet
        e->_prev = s->_tail;
        if (s->_tail != NULL) s->_tail->_next = e;
        s->_tail = e;
        if (s->_head == NULL) s->_head = e;
      }
    }
    // Increment the number of elements
    ++(s->_nbElem);
  }
}

// Function to insert an element pointing toward 'data' at the 
// 'iElem'-th position 
// If 'iElem' is greater than or equal to the number of element
// in the GSet, elements pointing toward null data are added
// Do nothing if arguments are invalid
void GSetInsert(GSet *s, void* data, int iElem) {
  // If the arguments are invalid, stop
  if (s == NULL || data == NULL || iElem < 0) return;
  // If iElem is greater than the number of elements, append
  // elements pointing toward null data to fill in the gap
  int nbAddElement = iElem - s->_nbElem - 1;
  while (nbAddElement > 0) {
    GSetAppend(s, NULL);
    nbAddElement--;
  }
  // If iElem is in the list of element or at the tail
  if (iElem <= s->_nbElem + 1) {
    // If the insert position is the head
    if (iElem == 0) {
      // Push the data
      GSetPush(s, data);
    // Else, if the insert position is the tail
    } else if (iElem == s->_nbElem + 1) {
      // Append data
      GSetAppend(s, data);
    // Else, the insert position is inside the list
    } else {
      // Allocate memory for the new element
      GSetElem *e = (GSetElem*)malloc(sizeof(GSetElem));
      // If we could allocate memory
      if (e != NULL) {
        // Memorize the pointer toward data
        e->_data = data;
        // By default set the sorting value to 0.0
        e->_sortVal = 0.0;
        // Set a pointer toward the head of the GSet
        GSetElem *p = s->_head;
        // Move the pointer to the iElem-th element
        for (int i = iElem; i > 0 && p != NULL; --i, p = p->_next);
        // Insert the element before the pointer
        e->_next = p;
        e->_prev = p->_prev;
        p->_prev = e;
        e->_prev->_next = e;
        // Increment the number of elements
        ++(s->_nbElem);
      }
    }
  }
}

// Function to insert an element pointing toward 'data' at the 
// tail of the GSet
// Do nothing if arguments are invalid
void GSetAppend(GSet *s, void* data) {
  // If the arguments are invalid, stop
  if (s == NULL) return;
  GSetElem *e = (GSetElem*)malloc(sizeof(GSetElem));
  if (e != NULL) {
    e->_data = data;
    e->_sortVal = 0.0;
    e->_prev = s->_tail;
    e->_next = NULL;
    if (s->_tail != NULL) s->_tail->_next = e;
    s->_tail = e;
    if (s->_head == NULL) s->_head = e;
    ++(s->_nbElem);
  }
}

// Function to remove the element at the head of the GSet
// Return the data pointed to by the removed element, or null if the 
// GSet is empty
// Return null if arguments are invalid
void* GSetPop(GSet *s) {
  // If the arguments are invalid, return null
  if (s == NULL) return NULL;
  void *ret = NULL;
  GSetElem *p = s->_head;
  if (p != NULL) {
    ret = p->_data;
    s->_head = p->_next;
    if (p->_next != NULL) p->_next->_prev = NULL;
    p->_next = NULL;
    p->_data = NULL;
    if (s->_tail == p) s->_tail = NULL;
    free(p);
    --(s->_nbElem);
  }
  return ret;
}

// Function to remove the element at the tail of the GSet
// Return the data pointed to by the removed element, or null if the 
// GSet is empty
// Return null if arguments are invalid
void* GSetDrop(GSet *s) {
  // If the arguments are invalid, return null
  if (s == NULL) return NULL;
  void *ret = NULL;
  GSetElem *p = s->_tail;
  if (p != NULL) {
    ret = p->_data;
    s->_tail = p->_prev;
    if (p->_prev != NULL) p->_prev->_next = NULL;
    p->_prev = NULL;
    p->_data = NULL;
    if (s->_head == p) s->_head = NULL;
    free(p);
    --(s->_nbElem);
  }
  return ret;
}

// Function to remove the element at the 'iElem'-th position of the GSet
// Return the data pointed to by the removed element
// Return null if arguments are invalid
void* GSetRemove(GSet *s, int iElem) {
  // If the arguments are invalid, return null
  if (s == NULL) return NULL;
  // Variable to memorize the return value
  void *ret = NULL;
  // If iElem is a valid index
  if (iElem >= 0 && iElem < s->_nbElem) {
    // Set a pointer to the head of the Gset
    GSetElem *p = s->_head;
    // Move the pointer to the iElem-th element
    for (int i = iElem; i > 0 && p != NULL; --i, p = p->_next);
    // Memorize the data at iElem-th position
    ret = p->_data;
    // Remove the element
    if (p->_next != NULL) p->_next->_prev = p->_prev;
    if (p->_prev != NULL) p->_prev->_next = p->_next;
    if (s->_head == p) s->_head = p->_next;
    if (s->_tail == p) s->_tail = p->_prev;
    p->_next = NULL;
    p->_prev = NULL;
    p->_data = NULL;
    free(p);
    // Decrement the number of elements
    --(s->_nbElem);
  }
  // Return the data
  return ret;
}

// Function to remove the first element of the GSet pointing to 'data'
// Do nothing if arguments are invalid
void GSetRemoveFirst(GSet *s, void *data) {
  // If the arguments are invalid, stop
  if (s == NULL) return;
  // Set a pointer toward the head of the GSet
  GSetElem *p = s->_head;
  // Loop on elements until we have found the 
  // requested data or reached the end of the list
  while (p != NULL && p->_data != data) {
    p = p->_next;
  }
  // If the pointer is null it means the data wasn't in the GSet
  if (p != NULL) {
    // Remove the element
    if (p->_next != NULL) p->_next->_prev = p->_prev;
    if (p->_prev != NULL) p->_prev->_next = p->_next;
    if (s->_head == p) s->_head = p->_next;
    if (s->_tail == p) s->_tail = p->_prev;
    p->_next = NULL;
    p->_prev = NULL;
    p->_data = NULL;
    free(p);
    // Decrement the number of elements
    --(s->_nbElem);
  }
}

// Function to remove the last element of the GSet pointing to 'data'
// Do nothing if arguments are invalid
void GSetRemoveLast(GSet *s, void *data) {
  // If the arguments are invalid, stop
  if (s == NULL) return;
  // Set a pointer toward the tail of the GSet
  GSetElem *p = s->_tail;
  // Loop on elements until we have found the 
  // requested data or reached the head of the list
  while (p != NULL && p->_data != data) {
    p = p->_prev;
  }
  // If the pointer is null it means the data wasn't in the GSet
  if (p != NULL) {
    // Remove the element
    if (p->_next != NULL) p->_next->_prev = p->_prev;
    if (p->_prev != NULL) p->_prev->_next = p->_next;
    if (s->_head == p) s->_head = p->_next;
    if (s->_tail == p) s->_tail = p->_prev;
    p->_next = NULL;
    p->_prev = NULL;
    p->_data = NULL;
    free(p);
    // Decrement the number of elements
    --(s->_nbElem);
  }
}

// Function to remove all the selement of the GSet pointing to 'data'
// Do nothing if arguments are invalid
void GSetRemoveAll(GSet *s, void *data) {
  // If the arguments are invalid, stop
  if (s == NULL) return;
  // Set a pointer toward the tail of the GSet
  GSetElem *p = s->_tail;
  // Loop on elements until we reached the head of the list
  while (p != NULL) {
    // If the element points toward data
    if (p->_data == data) {
      // Memorize the previous element before deleting
      GSetElem *prev = p->_prev;
      // Remove the element
      if (p->_next != NULL) p->_next->_prev = p->_prev;
      if (p->_prev != NULL) p->_prev->_next = p->_next;
      if (s->_head == p) s->_head = p->_next;
      if (s->_tail == p) s->_tail = p->_prev;
      p->_next = NULL;
      p->_prev = NULL;
      p->_data = NULL;
      free(p);
      // Decrement the number of elements
      --(s->_nbElem);
      // Continue with previous element
      p = prev;
    // Else, the element doesn't point toward data
    } else {
      // Continue with previous element
      p = p->_prev;
    }
  }
}

// Function to get the element at the 'iElem'-th position of the GSet
// without removing it
// Return the data pointed to by the element
// Return null if arguments are invalid
void* GSetGet(GSet *s, int iElem) {
  // If the arguments are invalid, return null
  if (s == NULL) return NULL;
  // Set a pointer for the return value
  void *ret = NULL;
  // If iElem is a valid index
  if (iElem >= 0 && iElem < s->_nbElem) {
    // Set a pointer to the head of the GSet
    GSetElem *p = s->_head;
    // Move to the next element iElem times
    for (int i = iElem; i > 0 && p != NULL; --i, p = p->_next);
    // If the pointer is not null (in case the GSet is empty)
    if (p != NULL)
      // Memorize the data pointed to by the elem
      ret = p->_data;
  }
  // Return the element
  return ret;
}

// Function to get the index of the first element of the GSet
// which point to 'data'
// Return -1 if arguments are invalid or 'data' is not in the GSet
int GSetGetIndexFirst(GSet *s, void *data) {
  // If the arguments are invalid, return -1
  if (s == NULL) return -1;
  // Set a pointer toward the head of the GSet
  GSetElem *p = s->_head;
  // Set a variable to memorize index
  int index = 0;
  // Loop on elements until we have found the 
  // requested data or reached the end of the list
  while (p != NULL && p->_data != data) {
    ++index;
    p = p->_next;
  }
  // If the pointer is null it means the data wasn't in the GSet
  if (p == NULL)
    index = -1;
  // Return the index
  return index;
}

// Function to get the index of the last element of the GSet
// which point to 'data'
// Return -1 if arguments are invalid or 'data' is not in the GSet
int GSetGetIndexLast(GSet *s, void *data) {
  // If the arguments are invalid, return -1
  if (s == NULL) return -1;
  // Set a pointer toward the tail of the GSet
  GSetElem *p = s->_tail;
  // Set a variable to memorize index
  int index = s->_nbElem - 1;
  // Loop on elements until we have found the 
  // requested data or reached the head of the list
  while (p != NULL && p->_data != data) {
    --index;
    p = p->_prev;
  }
  // If the pointer is null it means the data wasn't in the GSet
  if (p == NULL)
    index = -1;
  // Return the index
  return index;
}
\end{lstlisting}
\end{ttfamily}
\end{scriptsize}

\section{Makefile}

\begin{scriptsize}
\begin{ttfamily}
\begin{lstlisting}
OPTIONS_DEBUG=-ggdb -g3 -Wall
OPTIONS_RELEASE=-O3
OPTIONS=$(OPTIONS_RELEASE)

all : gset

clean:
	rm *.o gset
	
gset : gset_main.o gset.o Makefile
	gcc gset_main.o gset.o $(OPTIONS) -o gset -lm 

gset_main.o : gset.h gset_main.c Makefile
	gcc -c gset_main.c $(OPTIONS)

gset.o : gset.c gset.h Makefile
	gcc -c gset.c $(OPTIONS)

\end{lstlisting}
\end{ttfamily}
\end{scriptsize}

\section{Usage}

\begin{scriptsize}
\begin{ttfamily}
\begin{lstlisting}
#include <stdlib.h>
#include <stdio.h>
#include "gset.h"

struct Test {
  int v;
};

void TestPrint(void *t, FILE *stream) {
  if (t == NULL) {
    fprintf(stream, "(null)");
  } else {
    fprintf(stream, "%d", ((struct Test*)t)->v);
  }
}

int main(int argc, char **argv) {
  GSet *theSet = GSetCreate();
  fprintf(stderr, "Created the set, nb elem : %d\n", theSet->_nbElem);
  struct Test data[4];
  for (int i = 0; i < 4; ++i) data[i].v = i;

  GSetPush(theSet, &(data[1]));
  GSetPush(theSet, &(data[3]));
  GSetPush(theSet, &(data[2]));
  fprintf(stderr, "Pushed [1,3,2], nb elem : %d\n", theSet->_nbElem);
  fprintf(stderr, "Print GSet:\n");
  GSetPrint(theSet, stdout, &TestPrint, (char*)", ");
  fprintf(stderr, "\n");

  fprintf(stderr, "Pop elements :\n");
  while (theSet->_nbElem > 0) {
    struct Test *p = (struct Test *)GSetPop(theSet);
    fprintf(stderr, "%d, ", p->v);
  }
  fprintf(stderr, "\n");

  GSetPush(theSet, &(data[1]));
  GSetPush(theSet, &(data[3]));
  GSetPush(theSet, &(data[2]));
  fprintf(stderr, "Push back and drop elements :\n");
  while (theSet->_nbElem > 0) {
    struct Test *p = (struct Test *)GSetDrop(theSet);
    fprintf(stderr, "%d, ", p->v);
  }
  fprintf(stderr, "\n");

  GSetAppend(theSet, &(data[1]));
  GSetAppend(theSet, &(data[3]));
  GSetAppend(theSet, &(data[2]));
  fprintf(stderr, "Append back and pop elements :\n");
  while (theSet->_nbElem > 0) {
    struct Test *p = (struct Test *)GSetPop(theSet);
    fprintf(stderr, "%d, ", p->v);
  }
  fprintf(stderr, "\n");

  GSetAppend(theSet, &(data[1]));
  GSetAppend(theSet, &(data[3]));
  GSetAppend(theSet, &(data[2]));
  fprintf(stderr, "Append back and drop elements :\n");
  while (theSet->_nbElem > 0) {
    struct Test *p = (struct Test *)GSetDrop(theSet);
    fprintf(stderr, "%d, ", p->v);
  }
  fprintf(stderr, "\n");

  GSetAddSort(theSet, &(data[2]), data[2].v);
  GSetAddSort(theSet, &(data[3]), data[3].v);
  GSetAddSort(theSet, &(data[1]), data[1].v);
  fprintf(stderr, "Add sort [2,3,1] and get elements :\n");
  for (int i = 0; i < theSet->_nbElem; ++i) {
    struct Test *p = (struct Test *)GSetGet(theSet, i);
    fprintf(stderr, "%d, ", p->v);
  }
  fprintf(stderr, "\n");

  GSetInsert(theSet, &(data[0]), 0);
  GSetInsert(theSet, &(data[0]), 2);
  GSetInsert(theSet, &(data[0]), 8);
  fprintf(stderr, "Insert 0 at 0, 2, 8 and get elements :\n");
  for (int i = 0; i < theSet->_nbElem; ++i) {
    struct Test *p = (struct Test *)GSetGet(theSet, i);
    TestPrint(p, stderr);
    fprintf(stderr, ", ");
  }
  fprintf(stderr, "\n");

  GSet *clone = GSetClone(theSet);
  fprintf(stderr, "Clone the set and print it:\n");
  GSetPrint(clone, stdout, &TestPrint, (char*)", ");
  fprintf(stderr, "\n");
  GSetFree(&clone);

  GSetRemove(theSet, 7);
  GSetRemove(theSet, 1);
  GSetRemove(theSet, 0);
  fprintf(stderr, "Remove at 7,1,0 and get elements :\n");
  for (int i = 0; i < theSet->_nbElem; ++i) {
    struct Test *p = (struct Test *)GSetGet(theSet, i);
    TestPrint(p, stderr);
    fprintf(stderr, ", ");
  }
  fprintf(stderr, "\n");

  fprintf(stderr, "Index of first null data : %d\n", 
    GSetGetIndexFirst(theSet, NULL));
  fprintf(stderr, "Index of last null data : %d\n", 
    GSetGetIndexLast(theSet, NULL));

  GSetRemoveAll(theSet, NULL);
  fprintf(stderr, "Delete all null and get elements :\n");
  for (int i = 0; i < theSet->_nbElem; ++i) {
    struct Test *p = (struct Test *)GSetGet(theSet, i);
    TestPrint(p, stderr);
    fprintf(stderr, ", ");
  }
  fprintf(stderr, "\n");

  GSetFree(&theSet);
}
\end{lstlisting}
\end{ttfamily}
\end{scriptsize}

Output:\\
\begin{scriptsize}
\begin{ttfamily}
\begin{lstlisting}
Created the set, nb elem : 0
Pushed [1,3,2], nb elem : 3
Print GSet:
2, 3, 1
Pop elements :
2, 3, 1, 
Push back and drop elements :
1, 3, 2, 
Append back and pop elements :
1, 3, 2, 
Append back and drop elements :
2, 3, 1, 
Add sort [2,3,1] and get elements :
3, 2, 1, 
Insert 0 at 0, 2, 8 and get elements :
0, 3, 0, 2, 1, (null), (null), 0, 
Clone the set and print it:
0, 3, 0, 2, 1, (null), (null), 0
Remove at 7,1,0 and get elements :
0, 2, 1, (null), (null), 
Index of first null data : 3
Index of last null data : 4
Delete all null and get elements :
0, 2, 1, 
\end{lstlisting}
\end{ttfamily}
\end{scriptsize}

\end{document}


